\chapter{Physics of Fluids}
\label{chapter physics}

The mechanics of fluids are governed by the partial differential equations (PDEs) known as the \textit{Incompressible Navier-Stokes Equations}, or in case of inviscid fluids, the \textit{Euler Equations}. This chapter explains the meaning and intuition behind these equations, which are key to designing and implementing numerical simulation algorithms.

\section{Vector Calculus}
The fluid equations are commonly written in the language of vector calculus. A brief introduction of the main concepts and operators involved is given in this chapter. 


\gapM

\textbf{Scalar Field}

\gapS

A \textit{scalar field} on $ \mathbb{R} ^3 $ is a mapping $\phi : \mathbb{R} ^3 \rightarrow \mathbb{R} $ from 3D cartesian coordinates to scalar values. Example scalar fields include fluid density, or pressure, where a scalar value can be sampled in each point of the 3D space.

\gapM

\textbf{Vector Field}

\gapS

A \textit{vector field} on $ \mathbb{R} ^3 $ is a mapping $\phi : \mathbb{R} ^3 \rightarrow \mathbb{R} ^3 $ from 3D cartesian coordinates to 3D vectors. A very commonly used vector field is the velocity field $\u$, which describes the direction and speed of the fluid's movement at each point in the 3D space


\gapM

\textbf{The grad}

\gapS

Given a scalar field $\phi : \mathbb{R} ^3 \rightarrow \mathbb{R} $, the \textit{gradient} or \textit{grad} of the field is a vector field written as $\nabla \phi$, and it is defined by:
\begin{equation*}
    \nabla \phi = 
    \left(
    \begin{aligned}
        \frac{\partial \phi}{\partial x} \\
        \frac{\partial \phi}{\partial y} \\
        \frac{\partial \phi}{\partial z}
    \end{aligned} \right)
\end{equation*} 
The grad of a scalar quantity $\phi$ represents the rate of change of $\phi$ across each dimension. Moreover, $\nabla \phi$ computes the direction of movement which causes the greatest increase of $\phi$. 

The $\nabla$ operator can also be extended to scaler fields of higher dimensions: let $\phi : \mathbb{R} ^N \rightarrow \mathbb{R} $ be an N-dimensional scalar field, $\nabla \phi$ is then defined as:
\begin{equation*}
    \nabla \phi \begin{pmatrix}
        x_1 \\
        x_2 \\
        \vdots \\
        x_N
    \end{pmatrix}
    =
    \left(
    \begin{aligned}
        \frac{\partial \phi}{\partial x_1} \\
        \frac{\partial \phi}{\partial x_2} \\
        \vdots~~ \\
        \frac{\partial \phi}{\partial x_N}
    \end{aligned} \right)
\end{equation*} 

\gapM

\textbf{The div}

\gapS

Given a vector field $\u : \mathbb{R} ^3 \rightarrow \mathbb{R} ^3$, the \textit{divergence} or \textit{div} of the field is a scalar field written as $\nabla \cdot \u$, and it is defined by:
$$
    \nabla \cdot \u 
    = \nabla \cdot 
    \begin{pmatrix}
        \u_x \\
        \u_y \\
        \u_z
    \end{pmatrix}
    =
    \frac{\partial \u_x}{\partial x} +  
    \frac{\partial \u_y}{\partial y} +
    \frac{\partial \u_z}{\partial z}
$$
If $\u$ is the velocity field, then the scalar field $\nabla \cdot \u$ represents the speed at which the fluid is expanding or shrinking at each 3D location. Thus, a velocity field that satisfies $\nabla \cdot \u = 0$ would keep the fluid in constant volume, which is how most fluids behave in the real world.


\gapM

\textbf{The curl}

\gapS

Given a vector field $\u : \mathbb{R} ^3 \rightarrow \mathbb{R} ^3$, the \textit{curl} of the field is a scalar field written as $\nabla \cross \u$, and it is defined by:
$$
    \nabla \cross \u = 
    \nabla \cross \begin{pmatrix}
        \u_x \\
        \u_y \\
        \u_z
    \end{pmatrix}
    =
    \left(
    \begin{aligned}
        \frac{\partial \u_z}{\partial y} - 
            \frac{\partial \u_y}{\partial z} \\
        \frac{\partial \u_x}{\partial z} - 
            \frac{\partial \u_z}{\partial x} \\
        \frac{\partial \u_y}{\partial x} - 
            \frac{\partial \u_x}{\partial y} \\
    \end{aligned} \right)
$$
Informally, the curl of the velocity field is a measure of the local rotation of the fluid. Though not directly used in the equations and algorithms presented in this project, it is at the heart of a different class of algorithms, called the vortex methods\cite{angelidis2005simulation}.


\gapM

\textbf{The Laplacian}

\gapS

The \textit{Laplacian} operator, written $\nabla \cdot \nabla$, is defined to be the divergence of the gradient. For scalar field $\phi$, it can be computed that:
$$
\nabla \cdot \nabla \phi = 
\frac{\partial ^2 \phi}{\partial x^2}+
\frac{\partial ^2 \phi}{\partial y^2}+
\frac{\partial ^2 \phi}{\partial z^2}
$$
The Laplacian describes the difference between the average value of $\phi$ in the neighborhood of a certain point and the value of $\phi$ at that point. As defined, this operator takes a scalar field and returns a scalar field. However, The Laplacian is also often extended to be applied to vector fields, where 
$$
\nabla \cdot \nabla \u =
    \begin{pmatrix}
        \nabla \cdot \nabla \u_x \\
        \nabla \cdot \nabla \u_y \\
        \nabla \cdot \nabla \u_z
    \end{pmatrix}
$$


\section{The Eulerian and Lagrangian Viewpoints}

For any physical quantity that represents some property of a fluid, the field of that quantity, either scalar or vector, could be constantly evolving as time passes. There are two different approaches to tracking this rate of change: the Eulerian viewpoint and the Lagrangian viewpoint.

The Eulerian viewpoint considers the time derivative of quantities at fixed locations in the 3D space. For a scalar field $\phi$ which varies through time, its \textit{Eulerian derivative} is simply $\dfrac{\partial \phi}{\partial t}$. To be more precise, the Eulerian derivative $\dfrac{\partial \phi}{\partial t}$, evaluated at point $\textbf{x}$, is the rate of change of $\phi$ of the fluid at the fixed position $\textbf{x}$, despite the fact that the fluid could be in motion. This has the immediate consequence that the concept of Eulerian derivative fails to capture the fact that physical quantities are carried around (i.e advected) by the fluid. 

The Lagrangian viewpoint, on the other hand, tracks the rates of changes of quantities as it moves along the velocity field $\u$. In this approach, for a scalar field $\phi$, its derivative with respect to time is written as $\dfrac{D\phi}{Dt}$, and defined to be
$$
\frac{D\phi}{Dt} = \frac{\partial\phi}{\partial t} + \nabla \phi \cdot \u
$$ 
This derivative, known as the \textit{Lagrangian derivative} or \textit{material derivative}, can be justified by treating the fluid as a collection of infinitesimal particles, each carrying some quantities and moving along the velocity field. At time $t$, for each particle $p$ with position $\textbf{x}$, the quantity of $\phi$ it carries is $\phi_p = \phi(t,\textbf{x}(p))$. The derivative with respect to $t$ of this term computes the rate of change of $\phi _p$:
$$
\begin{aligned}
    \frac{d}{dt} \phi_p
        &= \frac{d}{dt} \phi(t,\textbf{x}(t)) \\
        &= \frac{\partial \phi}{\partial t} + \nabla \phi \cdot \frac{d\textbf{x}}{dt} \\ 
        &= \frac{\partial \phi}{\partial t} + \nabla \phi \cdot \u \\
        &=\frac{D\phi}{Dt}
\end{aligned}
$$
Which is precisely the Lagrangian derivative.

When formalizing the Euler and Navier-Stokes equations, the Lagrangian derivative $\dfrac{D}{Dt}$ will be automatically extended to be applied to vector fields, where each component of the vector field is differentiated separately. This allows the term $\dfrac{D\u}{Dt}$ to be written, representing the acceleration of the infinitesimal fluid particles:
\begin{equation}
    \label{Du/Dt}
    \frac{D\u}{Dt} = \frac{\partial\u}{\partial t}
    + \begin{pmatrix}
       \nabla \u_x  \cdot \u\\
        \nabla \u_y \cdot \u\\
        \nabla \u_z \cdot \u
     \end{pmatrix}
\end{equation}








\section{The Euler and Navier-Stokes Equations}
\label{Euler N-S Eqns}

Using the previously defined notations, the Euler equations, which governs the motion of an incompressible and inviscid fluid
%under gravity as the only external force
, can be written as

\begin{equation}
    \tag{Euler Equations}
    \left \{
    \begin{aligned}
         \frac{D\u}{Dt}   &=   -\frac{\nabla p}{\rho} + \textbf{g} \\
         \nabla \cdot \u   &=   0
    \end{aligned} \right.
    \label{eqn:Euler Equations}
\end{equation} 
where $\u$ is the velocity field, $p$ is pressure, $\rho$ is the fluid's density, and $\textbf{g}$ the acceleration caused by an external force field (e.g gravity).

A generalized version of these equations is the famous incompressible Navier-Stokes equations, in which a term that describes viscosity is added:
\begin{equation}
    \tag{Navier-Stokes Equations}
    \left \{
    \begin{aligned}
         \frac{D\u}{Dt}   &=   -\frac{\nabla p}{\rho} + \textbf{g} + \nu \nabla \cdot \nabla \u \\
         \nabla \cdot \u  &=   0
    \end{aligned} \right.
    \label{eqn:Navier-Stokes Equations}
\end{equation} 
where $\nu$ is the kinematic viscosity coefficient.


As described in the last section, the quantity $(\nabla \cdot \u)$ represents the rate at which the fluid is expanding or shrinking. Fluids in the real world usually remains in constant volume, unless in extreme conditions. This motivates the equation $\nabla \cdot \u = 0$, included in both Euler and Navier-Stokes.


Besides the incompressibility condition, both Euler and Navier-Stokes include another equation known as the momentum equation (which is in fact a set of equations, because the quantities are vectors). The momentum equation essentially specifies Newton's 2nd law: $\textbf{a}$ = $\dfrac{\textbf{F}}{m}$ , i.e the acceleration is the force divided by the mass.

As previously explained, the quantity $\dfrac{D\u}{Dt}$ represents the acceleration of the infinitesimal fluid particles. Thus, to explain the momentum equations, it remains to demonstrate that the right hand side correctly computes the force divided by the mass. Let the mass of the infinitesimal particle be $m$, and let the force be separated into the internal forces within the fluid $F_{in}$ and the external forces $F_{ext}$:
$$
\dfrac{D\u}{Dt} = \frac{F_{in} + F_{ext}}{m}
$$
With $\textbf{g}$ representing the acceleration caused by an external force field (e.g gravity), this can be rewritten as
$$
\dfrac{D\u}{Dt} = \frac{F_{in}}{m} + \textbf{g}
$$
The internal forces within a fluid is caused by an imbalance in pressure. Specifically, if one side of a infinitesimal particle has a greater pressure than the opposite side, then the particle will be pushed towards the low pressure region. This justifies why the pressure forces are in the direction of $-\nabla p$, which computes the direction of fastest decrease of pressure. It can be shown that this actual pressure force exerted on the particle is the negative pressure gradient $-\nabla p$ multiplied by its volume $V$, which gives
$$
\dfrac{D\u}{Dt} = -\frac{V \nabla p}{m} + \textbf{g}
$$
Using $\rho = \dfrac{m}{V}$, this becomes:
$$
\dfrac{D\u}{Dt} =  -\frac{\nabla p}{\rho} + \textbf{g}
$$
Which is Euler's momentum equation. It is important to note that the justifications given in this section merely offers intuitions, and is far from a rigorous mathematical derivation, which would not fit into this report due to its complexity. 


The Navier-Stokes momentum equation extends the Euler momentum equation by considering viscosity. In a viscous fluid, the velocity of a particle tends to diffuse into its surrounding particles, causing the velocity in the neighborhood to converge into its average. The difference between the average of $\u$ in the neighborhood and the value of $\u$ of the particle is captured by the Laplacian of the velocity: $\nabla \cdot \nabla \u$, thus adding a positive multiple of this quantity creates a viscous effect:
$$
\frac{D\u}{Dt}   =   -\frac{\nabla p}{\rho} + \textbf{g} + \nu \nabla \cdot \nabla \u 
$$
where $\nu$ is a constant property, known as the kinematic viscosity of the fluid. For water, which is a rather inviscid fluid, this quantity is negligible, at least for rendering purposes. When simulating water, considering the effects of viscosity requires considerable extra computation, while bringing little improvements to the visual fidelity. As a result, this project chooses to only solve the Euler equations during simulation. 


\section{Boundary Conditions}
\label{section boundary conditions}
For a fluid region that is not the entirety of $\mathbb{R}^3$, boundary conditions must be specified, which define the behavior of the fluid in the physical boundaries of the fluid region.

When simulating liquids, there are two types of boundary conditions: the solid boundaries and the free surface boundaries. At a solid boundary, the condition is 
$$
    \u \cdot \textbf{n} = 0
$$
where $\textbf{n}$ is the normal of the solid surface. This condition ensures that the fluid cannot flow into a solid. 

The second type of boundary is the free surface boundary, which is the boundary between the liquids and some region of space that is not occupied by anything. In this case, that region of space will not exert any force, and therefore pressure, to the fluid, which motivates the condition
$$
p = 0
$$
This free surface condition can be also applied to the boundary between liquid and air, which is because air is significantly lighter than liquid, and hence does not influence the motion of the liquid.

The liquid simulated in this project is contained within a cubic box. Moreover,it doesn't fill the box entirely and thus has a free surface. Thus, both type of boundary conditions must be considered during the simulation. 

\section{Multiple Fluids}
\label{section multiple fluids}
Finally, this section introduces an equation that governs the concentration changes in a mixture of more than one type of miscible fluids, for example, ink and water. In physics, the different types of fluids are sometimes referred to as \textit{phases}.

Concentration of different fluid phases will be represented using \textit{volume fractions}. Specifically, for an infinitesimal fluid element at location $\textbf{x}$ with volume $\textbf{V}(\textbf{x})$, and let the portion of this volume occupied by the $i$th fluids phase be $\textbf{V}^i(\textbf{x})$, then the concentration of the $i$th phase at $\textbf{x}$ will be $\alpha^i(\textbf{x}) = \dfrac{\textbf{V}^i(\textbf{x})}{\textbf{V}(\textbf{x})}$. Together, the $\alpha^i$ for all phases form a vector $\bm{\alpha}$, whose components are non-negative and sum up to $1$.

The diffusion among multiple fluid phases is a result of the random Brownian motion of the fluid particles. However, it's possible to model this process from a macroscopic viewpoint, where an equation can be written that governs the expectation of how the concentrations change:
\begin{equation}
    \tag{Advection-Diffusion Equation}
    \frac{D \alpha^i}{D t} = C\nabla \cdot \nabla \alpha^i
\end{equation}
where $C$ is the diffusion coefficient. Informally, if the concentration of a fluid region has a different concentration than its surroundings, this difference tends to "diffuse" into the neighborhood, and thus be reduced. This explains why the rate of change of the concentration is proportional to its Laplacian.