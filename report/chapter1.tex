\chapter{Introduction}

\section{Motivation}
% are these words too informal?
Fluids can be seen everywhere. The smoke coming out of chimney spreading in the wind, the milk in a cup mixing with the coffee, the calm flow of a river with tiny ripples under the rain, and the enormous waves of the ocean splashing onto the surface. Many of these these phenomenons have interesting and even stunning visual effects, thus, quite often, plausible images of these fluids need to be computationally generated for purposes such as cinematics and video gaming. 

% Is the hours->seconds thing appropriate? 
Due to the complexity underlying the behavior of fluids, accurate numerical simulation of fluids often require a huge amount of computational resources. In areas such as aeronautic engineering, where the importance of accuracy completely overrides that of efficiency, hours of CPU time can be spent on the simulation of a few seconds of fluid motion. Real time computer graphics applications, such as video games, usually do not require this level of accuracy, but instead asks the simulation to be computed in roughly the same amount of time as the physical process it represents. This efficiency requirement is met with the help of modern GPUs, which have massively parallel computing abilities. This project implemented three algorithms on GPUs: FLIP (Fluid-Implicit-Particle), PBF (Position Based Fluids), and PCISPH (Predicative-Corrective Incompressible Smoothed Particle Hydrodynamics), all of which can perform simulations in real time.

%photorealistically??
For computer graphics applications, fluid simulation isn't the only task. It is equally necessary for the software to display the results of simulation (i.e, the shape and motion of the fluid) to the user. This is especially important for interactive video games, where the user experience highly depends on the quality of the rendering, maybe even more so than the accuracy of simulation. This project focuses on the rendering of a special, but likely most important, type of fluid: liquids. A real time GPU program was implemented that photorealistically renders the shape and motion of liquids, and captures important optical phenomenons such as reflection, refraction, and attenuation of light within a colored liquid.

\section{Related Work}


Einstein's paper: \cite{Einstein}
